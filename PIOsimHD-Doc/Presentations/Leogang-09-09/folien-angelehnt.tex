\documentclass{beamer}
%\documentclass[notes=show]{beamer}

\usepackage[utf8]{inputenc}
\usepackage[ngerman]{babel}
\usepackage{rotating}
\usepackage{float}
\usepackage{subfigure}

\usepackage{latexsym}
\usepackage{floatflt}
\usepackage{graphicx}
%\definecolor{lightblue}{rgb}{0.8,0.9,1.0}
%\usetheme{PaloAlto}
%\usetheme{Darmstadt}
\usetheme{Frankfurt}
%\usetheme{Berlin}

%\usecolortheme{crane}
%\usecolortheme{wolverine}
\usecolortheme{seagull}
%\usecolortheme{albatross}
%\usecolortheme{beetle}
%\usecolortheme{lily}

\usepackage{pgf,pgfarrows,pgfnodes,pgfautomata,pgfheaps}
\usepackage{amsmath,amssymb}

\setbeamercovered{dynamic}


%notes:
\setbeamertemplate{note page}[plain] 
%compressed

\title[]{  Simulation paralleler E/A auf Anwendungs- und Systemebene }
\author{ \underline{Julian M. Kunkel} } %Research Group:
\institute{ Institut für Informatik \\  Parallele und Verteilte Systeme \\ Ruprecht-Karls-Universit\"at Heidelberg}
\date{  27.09.2009 }
\beamertemplatetransparentcovereddynamic
\setbeamerfont{note page}{size=\small}
%\setbeamersize{text margin left=0.1cm,text margin right=0.1cm}
%\renewcommand{\>}{\rangle}
%\newcommand{\<}{\langle} 
%\usebeamercolor[fg]{[page number]}

\setbeamertemplate{navigation symbols}{}
%\setbeamertemplate{note page}{}

\setbeamersize{text margin left=0.2cm}
\setbeamersize{text margin right=0.2cm}
\setbeamersize{sidebar width right=0cm}
\setbeamersize{sidebar width left=0cm}

\setbeamertemplate{footline}
{
\begin{beamercolorbox}{title}
\hspace*{0.2cm}
% \copyright 
Julian M. Kunkel 
\hfill
 \insertslidenavigationsymbol
 \insertframenavigationsymbol
 \insertsubsectionnavigationsymbol
 \insertsectionnavigationsymbol
 \insertdocnavigationsymbol
 \insertbackfindforwardnavigationsymbol
 \hfill\insertframenumber/\inserttotalframenumber
 \hspace*{0.2cm}
\end{beamercolorbox}
}

%\newcommand{\markNote}[1]{	\textsuperscript{\tiny [#1]} }
\newcommand{\markNote}[1]{	 }

%\AtBeginSection[]{
%\frame{
      %\frametitle{Outline}
	%\tableofcontents[current,hideallsubsections]
%	}
%}

\begin{document}
\frame{
	\titlepage
}

\frame{
\frametitle{Agenda}
\tableofcontents[hideallsubsections]
}

\newcommand {\myframe}[2]{
\frame{
	\frametitle{#1} 
	\begin{itemize}
	#2 
	\end{itemize}
}
}

\newcommand {\itemm}[1]{
  	\begin{itemize}
	\item #1
	\end{itemize}
}

\section{Angelehnte Projekte}

\subsection{Tracing}
\frame{
  \begin{block}{}
  \centering
  Tracing Umgebung
  \end{block}
}

\myframe{Tracing}{
\item Existierende Formate stellten nicht genügend Information bereit
\itemm{Simulator benötigt alle Parameter eines MPI Aufrufs
  \item Nichtzusammenhängende I/O muss gespeichert werden}
\item Eigene Tracing Umgebung
\itemm{
  Trace Format(e)
  \item MPIWrapper
  \item Viewer
  }
}

\subsection{Simulation von Energiesparstrategien}

\frame{
  \begin{block}{}
  \centering
  Simulation von Energiesparstrategien
  \end{block}
}

\myframe{Ziele}{
  \item Vergleiche (simulierten) Energieverbrauch der Anwendung mit verschiedenen Sparstrategien
  \item Strategien könnten auch (bedingt) Umstrukturierung des Programs berücksichtigen
  \itemm{bspw. verzögern von Netzwerk- und I/O Aktivität}
  \item Berechne minimale Energiekosten bei optimal (energiesparenden) Komponenten
  \itemm{d.h. Energieverbrauch ist proportional zur Leistung 
    \item Obere Schranke für Energieeinsparpotenzial 
  }
  \item Anwendbarkeit bei beliebigen Programmen
}

\myframe{Simulation von Energiesparstrategien - Methodik}{  
  \item Auslastung von CPU, Netwerk, I/O-subsystemen periodisch tracen
  \item Schätze Energieverbrauch der Komponenten basierend auf Auslastung
    \itemm{Lineare Interpolierung des Energieverbrauchs - Min/Max Auslastung}
  \item Bei Idle-Zeiten Stromsparmechanismen aktivieren
    \itemm{ACPI Model für Energieverbrauch/Dauer des Zustandwechsels}
  \item Strategien kennen künftige Last (im Trace) und steuern Stromsparmechanismen
  \itemm{Eine Umordung/Verschiebung der Last ist teilweise möglich}
  \item Vergleich der Simulierten Ergebnisse mit gemessenem Energieverbrauch
  \itemm{Ebenfalls eine Eichung des Models}
}

\myframe{Beispielstrategie - Approach Strategie}{
  \item Betrachte die Last der nächsten (zukünftigen) k-Intervalle
  \item Kosten Zustandswechsel und Aktivierung vs. Einsparungspotenzial
  \itemm{$\Rightarrow$ minimale Zeit für die es sich lohnt die Komponente in Sparmodus zu versetzen}
  \item Bei wenig Last $\le 10\%$  verzögere diese (wenn möglich) auf später
  \itemm{Bspw. 10 Intervalle a 10\% resultiert in einem Intervall mit 100\% Last}
}

\myframe{Ergebnisse - Jacobi PDE}{
 \item Iterationen: 5 (Rechenintensiv), 500 (Kommunikationsintensiv)
 \item Für I/O wurde jede Iteration ein Checkpointing auf PVFS gemacht
 \item Die Abweichung zwischen gemessener Leistungsaufnahme und Simulierter beträgt im Beispiel $\leq$ 3\%.
  \begin{table}
  \begin{center}
  \begin{tabular}{l||l|l|l|l}
  Konfiguration & Simple S. & Optimale S. & Approach S. & Eff. Geräte \\
  \hline
  \hline
  4 p/500 & 90\,Watt & 17\% & 19\% & 33\% \\
  4x2 p/500 & 70\,Watt & 2.3\% & 2.6\% & 9.3\% \\
  4 p/5 + I/O & 20\,Watt & 7\% & 11\% & 28\% \\
  4x2 p/5 + I/O & 18.5\,Watt & 4.7\% & 6.7\% & 23.5\% \\  
  \end{tabular}
  \end{center}
  \end{table}
  % 500 iterationen => 4 nodes x 2 prozesse
  % 5 iterationen
}

\myframe{Fazit}{
  \item Hohe Auslastung $\Rightarrow$ gute Energieeffizienz
  \item Potenzial für Energieeinsparung durch Lastungleichheit etc. vorhanden
  \item Simulation ist ein Ansatz für Berechnung
  \item Abschätzung für Energiekosten und Verlust durch Programme
}
\end{document}
