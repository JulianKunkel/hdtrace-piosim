%%%%%%%%%%%%%%%%%%%%%%% file template.tex %%%%%%%%%%%%%%%%%%%%%%%%%
%
% This is a general template file for the LaTeX package SVJour3
% for Springer journals.          Springer Heidelberg 2010/09/16
%
% Copy it to a new file with a new name and use it as the basis
% for your article. Delete % signs as needed.
%
% This template includes a few options for different layouts and
% content for various journals. Please consult a previous issue of
% your journal as needed.
%
%%%%%%%%%%%%%%%%%%%%%%%%%%%%%%%%%%%%%%%%%%%%%%%%%%%%%%%%%%%%%%%%%%%
%
% First comes an example EPS file -- just ignore it and
% proceed on the \documentclass line
% your LaTeX will extract the file if required
\begin{filecontents*}{example.eps}
%!PS-Adobe-3.0 EPSF-3.0
%%BoundingBox: 19 19 221 221
%%CreationDate: Mon Sep 29 1997
%%Creator: programmed by hand (JK)
%%EndComments
gsave
newpath
  20 20 moveto
  20 220 lineto
  220 220 lineto
  220 20 lineto
closepath
2 setlinewidth
gsave
  .4 setgray fill
grestore
stroke
grestore
\end{filecontents*}
%
\RequirePackage{fix-cm}
%
%\documentclass{svjour3}                     % onecolumn (standard format)
%\documentclass[smallcondensed]{svjour3}     % onecolumn (ditto)
%\documentclass[smallextended]{svjour3}       % onecolumn (second format)
\documentclass[twocolumn]{svjour3}          % twocolumn
%
\smartqed  % flush right qed marks, e.g. at end of proof
%
\usepackage{graphicx}
\usepackage{listings}
%
% \usepackage{mathptmx}      % use Times fonts if available on your TeX system
%
% insert here the call for the packages your document requires
%\usepackage{latexsym}
% etc.
%
% please place your own definitions here and don't use \def but
% \newcommand{}{}
%
% Insert the name of "your journal" with
% \journalname{myjournal}
%
\begin{document}

\title{HDReplay%\thanks{Grants or other notes
%about the article that should go on the front page should be
%placed here. General acknowledgments should be placed at the end of the article.}
}
\subtitle{Replay MPI-Communication with HDReplay}

%\titlerunning{Short form of title}        % if too long for running head

\author{Johann Weging \and Julian Kunkel \and Michael Kuhn \and Thomas Ludwig}

%\authorrunning{Short form of author list} % if too long for running head

\institute{F. Author \at
              first address \\
              Tel.: +123-45-678910\\
              Fax: +123-45-678910\\
              \email{fauthor@example.com}           %  \\
%             \emph{Present address:} of F. Author  %  if needed
           \and
           S. Author \at
              second address
}

\date{Received: date / Accepted: date}
% The correct dates will be entered by the editor


\maketitle

\begin{abstract}
Insert your abstract here. Include keywords, PACS and mathematical
subject classification numbers as needed.
\keywords{First keyword \and Second keyword \and More}
% \PACS{PACS code1 \and PACS code2 \and more}
% \subclass{MSC code1 \and MSC code2 \and more}
\end{abstract}

\section{Motivation}
\label{motivation}
The communication is one of big time consuming factors of MPI applications.
Tracing the communication is one of the most common ways to visualize it,
find certain patterns and delays. % ?
PIOsimHD is a toolset that provides tracing and writes out the traces in files,
simulating a virtual environment for real MPI applications and visualizing the
communication. HDReplay adds the functionality of replaying the communication of
a traced MPI application by replaying the trace file. This gives you the
possibility to replay the communication and simulate the computation on
different hardware without recompiling the application.



\subsection{Tracing MPI Applications}
\label{tracing}
The trace file is generated by the HDMPIwrapper, a library that can be used to
log MPI function calls which occur during the execution of a program. Every 
MPI-rank has its own trace file containing the executed MPI-functions.\\
The meta information about the MPI-Application are stored in a project file.

\subsection*{Gathering Meta Information}
The project file holds information about accessed files by MPI-I/O, the 
process topology, used communicators and the MPI-datatypes. The information
are stored in a XML-based format.\\[2ex]

\subsection{Keeping track of I/O and communicators}

\begin{itemize}
  \item concept of communicator ids 
  \item stored ranks 
\end{itemize}

\subsubsection{Trace file format}
The trace file is an XML-based format. One element of the the XML file contains
one MPI function call.

\begin{figure*}

\begin{lstlisting}
<Send size='4' count='1' tid='1' toRank='1' toTag='1' cid='0'  time='0.910993'
end='0.911039'/>
\end{lstlisting}

% figure caption is below the figure
\caption{Example of a traced MPI\_Send function call}
\label{fig:trace-example}       % Give a unique label
\end{figure*}

\begin{itemize}
  \item explanation of the parameters
  \item nested function
\end{itemize}



\section{HDReplay}
\label{intro-hdreplay}

HDReplay is a software tool which is capable of replaying the communication of an
MPI application. HDReplay reads the traces generated by HDMPIwrapper and replays
them using

\begin{itemize}
  \item explain in general what HDReplay is 
  \item note the MPI-Implementation independency by using the install MPI lib
  \item highlight the main parts of the workflow
  
  \begin{itemize}
    \item processing the project file
    \item working with communicators datatypes etc.
    \item preparing files for I/O
    \item replaying the trace file
    \item cleanup
  \end{itemize}
  
  \item figure of a flowchart visualizing the workflow
\end{itemize}

HDReplay will create the necessary files be for starting the replay if the
original file dose not exists or can not be accessed. HDReplay will create
temporary files with the same size as the original one and fill it with random
data.
 
\section{Related Work}

\begin{itemize}
  \item "A generic extension to OTF and its use for MPI replay"
  \item "Pattern Matching and I/O Replay for POSIX I/O in Parallel Programs"
  \item note the diffrence that these papers using replay for debugging propose
  \item no community platform 
\end{itemize}

\section{Paraweb}
Paraweb is a community website

\begin{itemize}
  \item "Community Plattform for Parabench"
  \item give some general information about paraweb and parabench
\end{itemize}




\section*{Performance Measuring and Comparison}
\label{performance}
This chapter will focus on the future work of HDReplay. In the previous chapter
Paraweb was introduced as a platform for scientists the distribute their trace
files. HDReplay will give the user the possibility to compare different traces 
to each other, this provides the comparison of different communication and I/O
pattern. But to provide this, there are some goals to accomplish. 

\subsection{Performance Measuring}
\begin{itemize}
  \item measuring the overall computed flops 
  \item other clusters will run the same time and compute more or less folps
  \item using likwid for measuring
  \item cooperating between likwid and HDMPIwrapper
\end{itemize}

\subsection{Simulating Computation Time} 

\begin{itemize}
  \item compute for the time between the MPI-function-calls
  \item computation in a extra thread that will be stopped and continued
    through singnals
  \item user can choose between different mathematical functions to get closer 
  to the performance of the real program
  \item using blas for different functionality
  \item computing random data
\end{itemize}

\subsection{Comparing Different Hardware}
\begin{itemize}
  \item tracing the original program
  \item replay the tracefile
  \item upload trace and measuring of the replay on paraweb
  \item other user download the trace 
  \item replay the trace and compare performance 
\end{itemize}

\subsection{Comparing a Real Application to a Trace File}

\begin{itemize}
  \item problem of the replay overhead
  \item configuring the hardware
  \item calculate the difference between real program and replay 
  \item using multiple programs to calculate the variation
  \item evaluate the method 
\end{itemize}

%% For one-column wide figures use
%\begin{figure}
%% Use the relevant command to insert your figure file.
%% For example, with the graphicx package use
%  \includegraphics{example.eps}
%% figure caption is below the figure
%\caption{Please write your figure caption here}
%\label{fig:1}       % Give a unique label
%\end{figure}
%%
%% For two-column wide figures use
%\begin{figure*}
%% Use the relevant command to insert your figure file.
%% For example, with the graphicx package use
%  \includegraphics[width=0.75\textwidth]{example.eps}
%% figure caption is below the figure
%\caption{Please write your figure caption here}
%\label{fig:2}       % Give a unique label
%\end{figure*}
%%
%% For tables use
%\begin{table}
%% table caption is above the table
%\caption{Please write your table caption here}
%\label{tab:1}       % Give a unique label
%% For LaTeX tables use
%\begin{tabular}{lll}
%\hline\noalign{\smallskip}
%first & second & third  \\
%\noalign{\smallskip}\hline\noalign{\smallskip}
%number & number & number \\
%number & number & number \\
%\noalign{\smallskip}\hline
%\end{tabular}
%\end{table}


%%\begin{acknowledgements}
%%If you'd like to thank anyone, place your comments here
%%and remove the percent signs.
%%\end{acknowledgements}

%% BibTeX users please use one of
%%\bibliographystyle{spbasic}      % basic style, author-year citations
%%\bibliographystyle{spmpsci}      % mathematics and physical sciences
%%\bibliographystyle{spphys}       % APS-like style for physics
%%\bibliography{}   % name your BibTeX data base

%% Non-BibTeX users please use
%\begin{thebibliography}{}
%%
%% and use \bibitem to create references. Consult the Instructions
%% for authors for reference list style.
%%
%\bibitem{RefJ}
%% Format for Journal Reference
%Author, Article title, Journal, Volume, page numbers (year)
%% Format for books
%\bibitem{RefB}
%Author, Book title, page numbers. Publisher, place (year)
%% etc
%\end{thebibliography}

\end{document}
% end of file template.tex

